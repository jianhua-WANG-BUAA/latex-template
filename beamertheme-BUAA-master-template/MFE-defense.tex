% !TEX TS-program = xelatex
% !Mode:: "TeX:UTF-8"
\documentclass[table,aspectratio=1610]{beamer}%for the projection version
%\documentclass[table,aspectratio=1610,handout]{beamer}%for the publish version
\usetheme{BUAA}
%\usepackage{ctex}
\usepackage{algorithm}
\usepackage{algorithmic}

%\usepackage[table]{xcolor}
\usepackage{tcolorbox}
\usepackage{tikz}
\usepackage{colortbl}

\usepackage{caption}
\setbeamertemplate{caption}[numbered]
\captionsetup[figure]{labelfont={bf},name={Fig.},labelsep=period}
\usepackage{ragged2e}

\DeclareMathOperator{\sgn}{sgn}
\usepackage{gensymb}%for the symbol degree with \degree
\renewcommand\arraystretch{1.5}

%set the main color of the beamer
\definecolor{buaa-blue-light}{HTML}{5A93E9}
\definecolor{buaa-blue-dark}{HTML}{005BAC}
\definecolor{buaa-blue}{HTML}{15499A}
\colorlet{main}{buaa-blue!80!black}

%set the background color of the beamer
\definecolor{buaa-white}{HTML}{E0ECF7}
\colorlet{background}{buaa-white!50}  


%%test the other color
\definecolor{main}{HTML}{182936}
\definecolor{background}{HTML}{E6E6E6}

\graphicspath{{./figures/},{./figs-thesis/}}

\title[the abbrivated title of the presentation]{{\normalsize \textcolor{separator}{objective of the PPT}}\\The title of the presentation}
\author{
    Last name first name \quad student number \\ Supervisor: name
}
\institute[BUAA]{School of XXXXX \\ Beihang University}
%\date{\today}
%\background{background.jpeg}

\begin{document}

%\maketitle
\begin{frame}[plain]%to ensure the titlepage is without headline and footline.
	\titlepage
%	\begin{figure}
%		\raggedleft
%		\includegraphics[height=0.1\textheight]{buaalogo.png}
%	\end{figure}
\end{frame}


\AtBeginSection[] % to show the current section at the beginning of each section.
{
	\begin{frame}[plain]{Outline}
	\tableofcontents[currentsection,currentsubsection]
	\end{frame}
}



\section{Introduction}

\begin{frame}{About the company}

	\begin{figure}
		\raggedright
		\includegraphics[width=0.15\textwidth]{logo.png}
	\end{figure}

	Name of the company Co.,Ltd. 

	\begin{enumerate}
		\item well-kn supplof on-e \alert{rapiection} solutions;
		\item a complete \alert{R\&D}, produfter-sales see system;
		\item ser the \alert{pubafety}, food saand ml safety industries.
	\end{enumerate}

	\begin{columns}
		\column[c]{.40\textwidth}
		\begin{figure}
			\centering
			\includegraphics[width=.8\textwidth]{image-test.png}
			\caption{\alert{Holeld} exsive deter.}
		\end{figure}

		\column[c]{.40\textwidth}
		\begin{figure}
			\centering
			\includegraphics[width=.8\textwidth]{image-test.png}
			\caption{\alert{Moile} expive dettor.}
		\end{figure}
	\end{columns}
\end{frame}

\begin{frame}{Content of the intership}	
	Norm (mathematics) From Wikipedia, the free encyclopedia 

	\begin{columns}
		\column[c]{.65\textwidth}		
			\begin{enumerate}[<+->]
%				\justifying
				\item In mathematics, a norm is a \alert{function};
				
				\item  form of the triangle \alert{inequality}, and is;
				
				\item two properties of a norm, but may ;
				
				\item g, obeys a form of the triangle ine;
			\end{enumerate}
		
		\column[c]{.35\textwidth}
			\begin{figure}
				\centering
				\includegraphics[width=\textwidth]{image-test.png}
				\caption{\alert{UAV} exsive detion.}
			\end{figure}
	\end{columns}
		
\end{frame}

\section{Research background}

\begin{frame}{QuadroAV modeling}
	\begin{columns}
		\column[c]{0.55\textwidth}
		Translational motion:
		{\footnotesize
			\begin{equation}\label{eq:model:translation}
			\begin{cases}
			m\ddot{X}^w = u_{c1}(\cos\phi \sin\theta \cos\psi) \\
			m\ddot{Y}^w = u_{c1}(\sin\phi \cos\psi ) \\
			m\ddot{Z}^w = mg - u_{c1} \cos\theta \\
			\end{cases}
			\end{equation}
		}
		
		Rotational motion:
		{\footnotesize
			\begin{equation}\label{eq:model:rotation}
			\begin{cases}
			I_{xx}\ddot{\phi} = u_{c2}l+\dot{\theta}\dot{\psi}(-I_{zz}) \\
			I_{yy}\ddot{\theta} = u_{c3}l+\dot{\phi}\dot{\psi}(I_{zz}-) \\
			I_{zz}\ddot{\psi} = u_{c4}+\dot{\theta}(I_{xx}-I_{yy}) \\
			\end{cases}
			\end{equation}
		}
		
		\column[c]{0.5\textwidth}
		\begin{figure}
			\centering
			\includegraphics[width=\textwidth]{image-test.png}
		\end{figure}
	\end{columns}
	
	\pause
	\begin{columns}
		\column[c]{0.55\textwidth}
		{\Large \color{red}
			\begin{equation*}
			\Downarrow
			\end{equation*}
		}
		Reduced model:
		{\footnotesize
			\begin{equation} \label{eq:system:continuous:initial}
			\begin{split}
			\dot{p}_i(t) & = v_i(t) \\
			\dot{v}_i(t) & = u_i(t)
			\end{split}
			\end{equation}
		}
		\pause
		\column[c]{0.1\textwidth}
		{\Large \color{red}
			\[
			\Longrightarrow
			\]
		}
		
		\column[c]{0.4\textwidth}			
		State space representaion:
		{\footnotesize
			\begin{equation} \label{eq:system:continuous}
			\dot{x}_i(t)=Ax_i(t)+Bu_i(t),
			\end{equation}
			{\normalsize where }
			$
			A=
			\begin{bmatrix}
			0  & 0  \\
			0  & 0  \\
			\end{bmatrix}
			, \
			B=
			\begin{bmatrix}
			0 \\
			0 \\
			\end{bmatrix}.
			$
		}
	\end{columns}
\end{frame}

\begin{frame}{Formation tracking}
	\begin{figure}
		\centering
		\includegraphics[width=0.6\textwidth]{image-test.png}
		\caption{Illustrof the ti forion tracking.}
	\end{figure}
\end{frame}



\section{Internsntent}

\subsection{Scenarstruction}

\subsection{Slid cool}
\begin{frame}{Sliding mode control}
	\begin{columns}
		\column[c]{.45\textwidth}
		\begin{figure}
			\centering
			\includegraphics[width=\textwidth]{image-test.png}
			\caption{Tratory of \alert{contnuous} system.}
		\end{figure}
		
		\pause
		\column[c]{.45\textwidth}
		\begin{figure}
			\centering
			\includegraphics[width=\textwidth]{image-test.png}
			\caption{Trajery of \alert{diete} system.}
		\end{figure}
	\end{columns}
\end{frame}


\begin{frame}{Discrete-time SMC protocol}
	The following \alert{disibuted} fortion procol is proposed, 
	
	\begin{equation}\label{eq:ui:final}
	\begin{split}
	u_i(k)=& (K\bar{B}(d_i+a_{i0}))^{-1} \bigg( K\bar{B}\sum a_{ij}u_j(k)-\Big[K\bar{A}\big(x_i(k)- \sum_{j=1, j\ne i}^{N} a_{ij}x_j(k) \big)  \\
	& -a_{i0}K\bar{A}x_0(k) -a_{i0}K\bar{B}\tilde{u}_{i0}(k) + \varepsilon T \sgn(s_i(k)) \\
	& -K\big((d_i+a_{i0})f_i(k+1)-\sum_{j=1, j\ne i}^{N} a_{ij} \big) \Big] \bigg)
	\end{split}
	\end{equation}
	where $\tilde{u}_{i0}(k)=\tilde{u}_1-\sgn(s_i(k))$, $\tilde{u}_1=(u_{min})/2$ and $\tilde{u}_2=(u_{max}-u_{min})/2$.
\end{frame}



\begin{frame}{Simulation results}
	\begin{columns}
		\column[c]{.45\textwidth}
		\begin{figure}
			\centering
			\includegraphics[width=\textwidth]{image-test.png}
			\caption{Trajecto of seven UAin 40s in experiment 1.}
		\end{figure}
		
		\column[c]{.45\textwidth}
		\begin{figure}
			\centering
			\includegraphics[width=\textwidth]{image-test.png}
			\caption{PositioAVs at 40s in experiment 1.}
		\end{figure}
	\end{columns}
\end{frame}

\subsection{Reincetrol}

\begin{frame}{Reinforcement learning}
	Markon Process (MDP) 
	
	\begin{figure}
		\centering
		\includegraphics[width=0.75\textwidth]{image-test.png}
		\caption{The agection inion process.}
	\end{figure}
\end{frame}


\begin{frame}{Reward function}	
	Denote $ d $ the distance between UAV and target.
	\rowcolors{1}{black!15}{black!03}%from the second row to alternate the row colors
	\begin{table}
		\small 
		\centering
		\caption{Reward function definition oracking task.}
		\label{table:reward:function}
		\begin{tabular}{|c|l|c|}
			\hline
			  & \bfseries Condition & \bfseries Reward ($ R $) \\ \hline
			1 & $ d<0.05m $ & +50  \\ \hline
			2 & $ d<0.05m $ & +50  \\ \hline
			3 & $ d<0.05m $ & +50  \\ \hline
			4 & $ d<0.05m $ & +50  \\ \hline
			5 & no other reward  & -1 \\ \hline	
		\end{tabular}
	\end{table}
	
\end{frame}



\begin{frame}{Dyna-Q algorithm}
	\begin{figure}
		\centering
		\includegraphics[width=.75\textwidth]{image-test.png}
%		\caption{Stage 2.}
	\end{figure}
\end{frame}

\begin{frame}{Training results}
	\begin{itemize}
		\item Stage 1: Fixedt tracking, with $ (x,y) =(1.2m,-1.2m) $ ;
		\item Stage 2: Randracking, with $ x , y \in [-3.6m,3.6m] $ ;
	\end{itemize}
	\begin{columns}
		\pause
		\column[c]{.59\textwidth}
		\begin{figure}
			\centering
			\includegraphics[width=\textwidth]{image-test.png}
			\caption{Stage 1.}
		\end{figure}
		
		\pause
		\column[c]{.41\textwidth}
		\begin{figure}
			\centering
			\includegraphics[width=\textwidth]{image-test.png}
			\caption{Stage 2.}
		\end{figure}
	\end{columns}
\end{frame}

\begin{frame}{Formatng wi RL}
	\begin{itemize}
		\item Leader UAV: \alert{circular} mot $ r=10m $;
		\item Followers: tracking the learealize a square formation;
	\end{itemize}
	\begin{columns}
		\pause 
		\column[c]{.4\textwidth}
		\begin{figure}
			\centering
			\includegraphics[width=\textwidth]{image-test.png}
			\caption{Five UAVs \alert{before} the formation.}
		\end{figure}
	
		\column[c]{.45\textwidth}
		\begin{figure}
			\centering
			\includegraphics[width=\textwidth]{image-test.png}
			\caption{Positions of five UAVs \alert{before} the formation.}
		\end{figure}
	\end{columns}
\end{frame}

\begin{frame}{Forma trackin RL}
	\begin{itemize}
		\item LeadeAV: \alert{ciular} mot with $ r=10m $;
		\item Follors: tng the lea realize a square formation;
	\end{itemize}
	\begin{columns}
		\pause
		\column[c]{.4\textwidth}
		\begin{figure}
			\centering
			\includegraphics[width=\textwidth]{image-test.png}
			\caption{Five UAVs \alert{after} the formation.}
		\end{figure}
		
		\column[c]{.45\textwidth}
		\begin{figure}
			\centering
			\includegraphics[width=\textwidth]{image-test.png}
			\caption{Positions of five UAVs \alert{after} the formation.}
		\end{figure}
	\end{columns}
\end{frame}

\section{Conclusion}

\begin{frame}{What I have donnterhip}
	\begin{itemize}
		\item Utilization of timulator;
		\item Survey on the sle control theory;
		\item Survey on throl method with mulUAV system;
		\item RL algorithm trainingntation onplatform;
	\end{itemize}

	\begin{figure}
		\centering
		\includegraphics[height=.25\textheight]{image-test.png} \hspace{2em}
		\includegraphics[height=.23\textheight]{image-test.png} \hspace{2em}
		\includegraphics[height=.25\textheight]{image-test.png}
	\end{figure}

\end{frame}


\begin{frame}{Self-evaluation}
	\begin{columns}
		\column[c]{.75\textwidth}
		\begin{itemize}
			\item What I have \alert{learned}
			\begin{itemize}
				\item Understanding of
				\item Ability to solve complexe problems;
				\item Quality of oral communication;
			\end{itemize}
		\vspace{2ex}
			\item To be \alert{improved}
			\begin{itemize}
				\item The ability to solve complems across fields and disciplines;
				\item The ability to integrate and n special environments;
				\item The ability to ask quality questions;
			\end{itemize}
		\end{itemize}
	
		\column[c]{.25\textwidth}
		\begin{figure}
			\centering
			\includegraphics[width=\textwidth]{image-test.png}
		\end{figure}
	\end{columns}
\end{frame}

\section*{Acknowledgements}
\begin{frame}
	\centering
	\Huge
	Thanks for your attention!\\
	Q\& A
\end{frame}

\end{document}
